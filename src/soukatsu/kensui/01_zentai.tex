\subsection*{全体総括}

\writtenBy{\kensuiChief}{宮寺}{大樹}
%\writtenBy{\kensuiStaff}{宮寺}{大樹}

2022年度春学期の研究推進局は以下の3点を目的として活動を行った.

\begin{itemize}
    \item 平常活動の支援
    \item 会員が興味関心のある活動ができる環境づくり
    \item 発信力を養うための環境づくり
\end{itemize}

\subsubsection*{平常活動の支援}
平常活動の支援に関しては,プロジェクト活動の進捗管理やサポートを行ったが,
追い込み合宿,プロジェクト発表会は行われなかった.

プロジェクト活動の進捗管理では,週報を用いてプロジェクト活動の進捗確認や問題の有無の確認を行い,
問題が確認された場合は,それを上回生会議の議題に上げることで問題の解決を図るという方針であったが,
2022年度春学期は週報の回答が2件のみとなっており,その結果活動状況の把握が全くできておらず
また,その影響で成果が出せなかったプロジェクトも多く存在している.
以上より,週報によるプロジェクト管理は機能していなかった.

2022年度春学期はプロジェクト活動をオンラインと対面のハイブリッドで行い,
部屋取りはTriRを用いて行われた.

追い込み合宿はプロジェクト活動が十分ではなかった為,行われなかった.

プロジェクト発表会は全四つのプロジェクトの内二つが通年,Linux班が通年への移行を検討している,
かつLT・プレゼン班が十分な活動をできていなかった為,実施できなかった.

2022年度春学期の活動では対面でのプロジェクト活動をすると,
班員の都合の比較的合いやすい19時以降の活動が出来ない為,結果として活動回数が減少することが判明した.

\subsubsection*{会員が興味関心のある活動ができる環境づくり}
会員が興味関心のある活動ができる環境づくりに関しては,
夏期勉強会の準備を行った.

夏期勉強会の準備は,自分が開催したい勉強会と,他の会員に開催してほしい勉強会を募集した.
勉強会は対面で開催され,部屋取りはTriRを用いて行った.

2022年度夏期は,以下に示す 1 つの勉強会が開催された.

\begin{itemize}
    \item Git勉強会
\end{itemize}


\subsubsection*{発信力を養うための環境づくり}
発信力を養うための環境づくりに関しては,LTを行った.

LTは,毎週の定例会議中で行われた.
2022年度春学期は,LTを担当週までに行うことができず,遅延して行われたものが数件あった.
理由としては,担当者のアナウンスがメールのみであり,毎回の LT 後のアナウンスや,Slack でのアナウンスを行っていなかった点や
7月7日,14日の担当者を直前に決定・告知した点,テスト直前の担当であった点が挙げられる.

また,LT意欲向上のため,LTアンケートを行い,入賞者には本会で購入する本の選択権を与えた.
2022年度春学期は,最後の定例会議での対面LTアンケートの実施はしたが,局長の失念によりSlackでの告知が行われず
十分な票数を得られなかった為,LTアンケートは実質行われなかった.
