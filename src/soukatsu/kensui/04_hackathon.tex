\subsection*{ハッカソン総括}

%\writtenBy{\kensuiChief}{宮寺}{大樹}
\writtenBy{\kensuiStaff}{Park}{Jooinh}

ハッカソン総括にあたり,2022年度に行われたハッカソンは夏期ハッカソンのみであったため,主に扱う内容は夏期ハッカソンのことである.

\subsubsection*{全体総括}

2022年度夏期ハッカソンは夏期休暇中の8月27日から8月29日の3日間開催された.参加者は12名であったため,3〜4名からなる3グループでの制作となった.

ハッカソン開催の目的及び目標として掲げていた制作活動の促進や技術力・発表力の向上,会員間の交流を深める
といったような点は達成できたが,参加人数の低迷は憂慮すべき点であると考えられる.特に\thirdGrade{}の参加人数が少なかった.

\subsubsection*{事前運営総括}

企画の告知においては,2022年度春学期の定例会議とSlackを通じて行った.

2022年度夏期ハッカソンは対面とオンライン両方で参加できるようにしていた.
エポック立命21の交流室と宿泊施設を予約しようとしたが,宿泊施設の申請が間に合わず,交流室の利用のみとなった.

また,オンライン参加の人員は例外なく失踪したため,対面ハッカソンを行う場合はオンライン参加を認めないか,
あるいはオンライン参加の班と対面参加の班で分ける必要があると考えられる.

開発ツールとしてGitとSlackを使い,開会式と閉会式はZoomを介して行った.
テーマに関しては開会式当日に発表し,アイデアソンを行った.

\subsubsection*{当日運営総括}

例年に比べて参加者の遅刻などが目立っていたため,スケジュールは多少変更された.

成果物発表では,発表スライドを作成した班と成果物を披露した班がいた.
どちらも発表はビームプロジェクターを使って行った.
\thirdGrade{}の参加が少なかったため,成果物を完成させた班はなかったが,ハッカソンの経験を積むという目標はある程度達成できた.
