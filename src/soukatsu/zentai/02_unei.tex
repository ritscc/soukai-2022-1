\subsection*{運営総括}

%\writtenBy{\president}{山本}{京介}
\writtenBy{\subPresident}{山本}{京介}
%\writtenBy{\firstGrade}{山本}{京介}
%\writtenBy{\secondGrade}{山本}{京介}
%\writtenBy{\thirdGrade}{山本}{京介}
%\writtenBy{\fourthGrade}{山本}{京介}

2022年度春学期の運営を以下の4点から述べる.
\begin{itemize}
    \item 定例会議
    \item 上回生会議
    \item 局
    \item 企画
\end{itemize}

\subsubsection*{定例会議}
毎週木曜日にZoomを用いて開催した.
2022年度は対面での開催が可能であったため,対面を主な開催形式とした.
しかし,対面での開催だと帰宅の時間が遅くなったり,そもそも大学に行くことが難しい会員がいたりしたため,Zoomによる配信も行った.
内容は例年通り執行部及び局からの連絡,会員によるLTであった.
定例会議で連絡するべき議題は上回生会議やSlackで事前に執行部で共有した.

定例会議を開催する教室は\president{}が予約していたが,失念することもあったため\kensuiChief{}も予約を行うこととなった.
Zoomの管理は\subPresident{}の担当となった.

formviewerに関しては,引き継ぎが十分ではなく前半はあまり使われなかった.
後半は積極的な使用を呼びかけ,使用頻度を上げていった.

\subsubsection*{上回生会議}
毎週月曜日にZoomを用いて開催した.
参加に関して,参加率は高かったが,欠席する場合に代理人を立てない場合が多かった.
企画担当者やプロジェクトリーダーの招集などの出席も問題なく参加できており,
\fourthGrade{}の参加者も見受けられた.
新入生に参加可能であることは呼びかけたが,参加は特になかった.
内容に関しては,議題の共有など十分できており,上回生のフォローもあり運営は円滑に行われた.

\subsubsection*{局}
局の運営では局会議と局配属にさらに分けて述べる.
\paragraph*{局会議}
そもそも局長だけで構成されている局が多く,局会議が行われなかった局が多かった.
その代わりに上回生会議で必要な議論は行った.
システム管理局は話すべき議題があったため局会議が開かれたが,毎週開催ではなかった.
\paragraph*{局配属}
希望調査が遅れてしまい,局配属は行えなかった.

\subsubsection*{企画}
基本的に担当者を2名設けるべきであったが,\secondGrade{}の人数が少なく,ほとんどの企画は担当が1名となった.
運営主体は\secondGrade{}であるが,必要に応じて\thirdGrade{}がサポートを行った.
企画担当者の上回生会議への出席は前述のように問題なく,企画の進捗は上回生会議で確認することができた.
しかし,KPTは一部の企画では行えていない.
