\subsection*{Welcomeゼミ総括}

%\writtenBy{\president}{古川}{聡悟}
%\writtenBy{\subPresident}{古川}{聡悟}
%\writtenBy{\firstGrade}{古川}{聡悟}
\writtenBy{\secondGrade}{古川}{聡悟}
%\writtenBy{\thirdGrade}{古川}{聡悟}
%\writtenBy{\fourthGrade}{古川}{聡悟}

2022年度も2021年度に引き続きWelcomeゼミを行った.
形式は2021年度同様,新入生の希望に応じて上回生との1対1の指導方式であったが,2022年度は対面で行われた.

上回生から担当できる分野を聞き出し,新入生には希望する分野を選んでもらいペアを組んだ.
新入生の希望分野はある程度分散されており,偏りは見受けられなかった.
本来\secondGrade{}と\thirdGrade{}のみで新入生を教育するべきであった.
しかしながら,その中の上回生で教えることができない分野も誤って募集したため,結果として,\fourthGrade{}も教える側になる必要性があった.
これは,上回生のスキルマップのアンケートを提示するのが遅れ,提出率が低くなったことが原因として挙げられる.
また,進捗報告については毎週日曜日にSlackで通知することで上回生が報告することができた.

また,2022年度は,2021年と同様に以下を目標として進行した.
\begin{itemize}
	\item 新入生にとっ本会のサークルルームが居心地のよい空間にする
	\item 気軽にサークルルームに来てもらう
	\item 新入生の中長期的な定着
\end{itemize}

サークルルーム関係の目標に関してはWelcomeゼミ以外で新入生が来ることが少なく,サークルルームの使い方が周知されておらず未達成であった.
また,新入生の中長期的な定着目標は,入会率が高かったため達成できたと言える.
さらに,成果物発表はWelcomeゼミに参加した新入生15人中9人が発表できたため積極的に参加できていた.
