\subsection*{新歓総括}

%\writtenBy{\president}{山本}{京介}
\writtenBy{\subPresident}{山本}{京介}
%\writtenBy{\firstGrade}{山本}{京介}
%\writtenBy{\secondGrade}{山本}{京介}
%\writtenBy{\thirdGrade}{山本}{京介}
%\writtenBy{\fourthGrade}{山本}{京介}

2022年度春学期の新歓の目的は,以下の2点であった.

\begin{itemize}
    \item 新入生に会の活動内容や活動方針について知ってもらう
    \item 新入生に会に興味を持ってもらう
\end{itemize}

これらの目的を達成するため,以下の4点の目標を掲げた.

\begin{itemize}
    \item 企画に対して参加してもらう
    \item 気軽にサークルルームに来てもらう
    \item 新入生に本会でやりたい事を見つけてもらう
    \item 新入生の中長期的な定着
\end{itemize}

これらの目標を達成するため,まず大学が主催する対面ブースに参加した.
対面ブースは主催を変えて複数回行われたが,できる限り参加した.
ブースでは活動の紹介や,サークルルームの場所の説明,後日開催の新歓団体企画への誘導を行った.

対面ブースの他に大学主催の新歓団体企画に応募,開催した.
新歓団体企画では活動紹介の他,LT会を開催した.
ブースやTwitterで告知が功を奏し,多くの新入生に参加してもらえた.
告知に関しては,他に大学に申請しビラを掲載しようと試みたが,大学からの応答がなく掲載は叶わなかった.

大学側が主催するイベント以外に二次企画として,上回生が新入生に何か一つ簡単なものを教えるという企画を行った.
類似の企画としてWelcomeゼミが挙げられるが,Welcomeゼミと違い1日である分野の入門だけ行うという形式を取った.
二次企画は基本的にサークルルームで行った.
二次企画によって新入生に本会でやりたい事を見つけてもらう他,気軽にサークルルームにきてもらうきっかけにもなったと考えられる.

以上の対面ブース,新歓団体企画,二次企画での活動により,2022年度における新歓の目的は達成されたと考えられる.
