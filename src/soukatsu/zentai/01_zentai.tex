\subsection*{前期活動総括}

\writtenBy{\president}{Park}{Jooinh}

本会の目的である「情報科学の研究,及びその成果の発表を活動の基本に会員相互の親睦を図り,学術文化の創造と発展に寄与する」ことを達成するため,方針として以下の六つを立てた.
これらについてそれぞれ評価を行うことで2021年度春学期の総括とする.

\begin{itemize}
  \item 親睦を深める
  \item 規律ある行動
  \item 自己発信力の向上
  \item 会員間の技術向上
  \item 外部への情報発信
  \item 持続可能な運営
\end{itemize}

\subsubsection*{親睦を深める}
2022年度春学期活動では,Welcomeゼミや新歓交流会,プロジェクト活動を実施することによって会員間の親睦を図った.

尚,2022年度春学期活動においては対面活動が許可されているため,活動形式は状況に応じてオンライン,対面,あるいは両方で行う形をとった.

Welcomeゼミは,新入生と上回生が親睦を深める重要な機会であった.
2022年度春学期では対面で行うことができたため,例年より親睦を深めることができたペアが多かった.

新歓交流会にも多くの新入生が参加し,自己紹介後には外部で食事会を行い交流の場を設けた.
上回生との会話を通して,良い影響があったように思われる.

プロジェクト活動では,各々の班で共同開発や発表が活発に行われた.
対面形式で行うことができた班もあったため,会員同士の交流の機会はあったと思われる.

\subsubsection*{規律ある行動}
本項では,遅刻・欠席連絡と備品整備,サークルルームの使用方法の三つについて評価する.

遅刻・欠席連絡については,概ねSlackの専用チャンネルにおいて行われていた.
例年に比べ連絡頻度が多く,正確な理由の記述が多くなった.
また開始時刻を過ぎてからの連絡は少なかった.

備品整理及びサークルルームの使用方法については,ビームプロジェクターの貸し出しがあったが,事前に許可をもらっていたので問題なく行われた.
また,サークルルームは二回生以上の利用が多く,一回生の利用はやや少なかった.

\subsubsection*{自己発信力の向上}
自己発信力の向上の機会として,2022年度春学期活動では,LTを行った.

LTは,割り当てられていた会員の全員が発表したため,定例会議での発表は充実していた.
しかし新入生の飛び入りLTは無く,また有志の飛び入りも多いとはいえなかった.積極的な会員のみが発表を行う状態であった.

プロジェクト発表会については,半期プロジェクトの班が報告書執筆と発表が間に合わず行われなかった.
代わりにそれらのプロジェクトを通年プロジェクトに切り替えて秋学期でプロジェクト発表会を行うこととする.

\subsubsection*{会員間の技術向上}
会全体の技術力を向上させることを目的として,LTやプロジェクト活動,勉強会を開催した.

LTの内容には簡単のものが多く,内容が新入生には伝わりやすかった.
しかし,会員が自身の興味対象について深い内容を扱ったものが少なかったことも否めない.
しかし新入生の発表は無く,また有志の飛び入りも多いとはいえなかった.

プロジェクト活動は対面で行ったが,進捗管理に問題があり,完了することはできなかった.
勉強会は対面で無事に行われた.

\subsubsection*{外部への情報発信}
会外へ活動を発信する機会として,主に本会Webサイトと会公式Twitter,KC3が挙げられる.

本会Webサイトの更新は2021年度春学期プロジェクト活動報告書のみであり,不足していた.新歓に関する告知やイベントに関しての記事を一切上げることができなかった.

会公式Twitterでは,LTやイベントが行われる度にその様子が発信された.
こちらは頻度が十分であり,内容も適切であった.

また,KC3では本会の活動紹介や勉強会開催を通して,本会の活動を発信した.
懇親会においては,複数の会員が参加し,自身の興味分野について発信した.

\subsubsection*{持続可能な運営}
本会はコロナ禍によって,多くの活動が縮小を迫られた.
そこで持続可能な運営にすべくサークルルームの利用再開と新入生の勧誘を挙げた.

まずサークルルームの利用再開については,
立命館大学のガイドラインに沿いつつ本会の実情に合わせた部の対応方針を作成し,学生部に提出した.
ここで複数回の再提出を経て,秋学期から制限付きでのサークルルーム利用再開,及び例会とプロジェクト活動の対面活動が許可された.
よってこの目標は達成することができたと思われる.

次に新入生の勧誘について,
ブースを複数回出展し,新歓企画にも参加した.
それにより入会を希望したであろう新入生を十分確保できたと思われるため目標を達成できた.
