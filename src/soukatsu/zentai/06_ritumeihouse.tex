\subsection*{立命の家総括}

%\writtenBy{\president}{羽田}{秀平}
%\writtenBy{\subPresident}{羽田}{秀平}
%\writtenBy{\firstGrade}{羽田}{秀平}
\writtenBy{\secondGrade}{羽田}{秀平}
%\writtenBy{\thirdGrade}{羽田}{秀平}
%\writtenBy{\fourthGrade}{羽田}{秀平}

\subsection*{概要}
立命の家は毎年立命館大学で開催される小学生を対象とした企画である.
2022年度は8月22日,8月23日の二日間開催された.

\subsection*{目的と目標}
\begin{itemize}
    \item 学術部公認団体として求められる還元活動の義務を果たすこと
    \item 本企画に参加する小学生に本会の活動と情報技術に興味・関心を向けてもらえること
    \item 小学生に教える体験を通して会員の教える能力の向上をはかり今後の還元活動を円滑にできるようになること
\end{itemize}
\subsection*{実施内容と提案}
2022年度は二日間にわたって開催され,1日目は対面,2日目はオンラインで行った.
Scratchを用いたプログラミング勉強会を行った.
2021年度は参加者に理解してもらうことが難しかったため,2022年度は参加者が作りたい機能を完成されたゲームに追加していくことを行った.

\subsection*{対象}
1日目は5年生以上推奨にし,2日目は6年生以上推奨にしたため企画を滞りなく行うことができた.
また2日とも参加人数を6まで減らした.

\subsection*{役割}
担当者2人で役割分担してうまく運営することができた.

\subsection*{改善点}
企画を行った班によっては,参加者が追加したい機能を担当者が考える間,参加者は待っていることが多く見受けられた.
確認不足で副担当が企画に参加できないことが直前で発覚した.
