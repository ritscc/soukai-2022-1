\subsection*{運営方針}

%\writtenBy{\president}{山本}{京介}
\writtenBy{\subPresident}{山本}{京介}
%\writtenBy{\firstGrade}{山本}{京介}
%\writtenBy{\secondGrade}{山本}{京介}
%\writtenBy{\thirdGrade}{山本}{京介}
%\writtenBy{\fourthGrade}{山本}{京介}

2022年度秋学期の運営に関して以下の5点から方針を述べる.
\begin{itemize}
    \item 定例会議
    \item 上回生会議
    \item 局
    \item 企画
    \item 運営サポート
\end{itemize}

\subsubsection*{定例会議}
春学期同様木曜日に対面中心で開催する.
必要に応じてSlackのチャンネルで資料や議題の周知と共有を行う.
会内向けのフォームに関しては,積極的にformviewerを活用する.

\subsubsection*{上回生会議}
秋学期も春学期同様に毎週開催する.
各局の局長またはその代理人が必ず参加するものとする.
各局長は局員が上回生会議における議題の内容を把握できるようにする.
また,例年通り議決権のない会員,特に\firstGrade{}に対して上回生会議に
参加可能である旨を告知する.

\subsubsection*{局}
局配属に関しては遅れが生じているため,早急に取り組む.
局会議に関しては局によって議題の量に差があるため,
2022年度秋学期では局会議の定期的な開催の強制をしないこととする.
ただし,上回生会議にて都度局での状況を確認し,必要に応じて局会議を開催する.

\subsubsection*{企画}
担当者は1名に対する負担を軽減するため,常に会員2名以上で対応することが望ましい.
担当者の一人が退会などの理由で対応できなくなった場合,
新たに担当者を追加するか,2021年度担当者がフォローできるようにする.
KPTに関しては,基本的に上回生会議で行うが,主催が局である場合は局会議で行う.

\subsubsection*{運営サポート}
春学期同様,\thirdGrade{}がサポートするが,春学期の状況などを顧み,より\secondGrade{}中心の運営を行っていく.
秋学期からは\thirdGrade{}が研究室に配属されるため,必要な時は可能な限り早めの連絡を心がける.
