\subsection*{プロジェクト活動総括}

\writtenBy{\kensuiChief}{宮寺}{大樹}

本項では本局におけるプロジェクト活動業務に関する2022年度春学期の総括を以下の点において述べる.

\begin{itemize}
\item 企画書の募集
\item 週報の回収・催促
\item 会員のプロジェクト管理
\item 発表の機会の提供
\item 報告書の管理
\end{itemize}

\subsubsection*{企画書の募集}

プロジェクト活動の企画書は最初の募集で三つ,追加募集で一つ提出され,最終的に四つ企画書が提出された.
企画書を局会議と上回生会議で確認を行い,全ての企画書に問題が無かったため,全てのプロジェクトが設立された.

\subsubsection*{週報の回収・催促}

各プロジェクトリーダーは,プロジェクト活動の進捗確認や問題の有無の確認を行うために,
週報の提出が義務付けられている.
週報の回収にはGoogleフォームが用いて行われたが,Slack のリマインダー機能を用いてのリマインドは行われなかった.
よって週報の提出は2件のみとなっていた.

\subsubsection*{会員のプロジェクト管理}

本局は,各会員がどのプロジェクトに所属しているかを把握し,
プロジェクトが途中で終了した場合などに所属していた会員のプロジェクト異動などを管理している.
プロジェクト間での会員の異動はなかった.

\subsubsection*{発表の機会の提供}

プロジェクト活動の成果発表をプロジェクト発表会を通じて行うこと予定していたが,.
4分の3のプロジェクトが通年のもしくは通年への移行を検討中かつLT・プレゼン班が
十分な活動をできていなかった為実施できなかった. 
