\subsection*{プロジェクト活動総括}

%\writtenBy{\president}{宮寺}{大樹}
%\writtenBy{\subPresident}{宮寺}{大樹}
%\writtenBy{\firstGrade}{宮寺}{大樹}
\writtenBy{\secondGrade}{宮寺}{大樹}
%\writtenBy{\thirdGrade}{宮寺}{大樹}
%\writtenBy{\fourthGrade}{宮寺}{大樹}

\subsubsection*{全体総括}
2022年度春学期のプロジェクト活動は,5月下旬頃から企画書の募集を開始し,6月中旬に活動を開始した.
各プロジェクトには活動ごとに週報を提出することを義務付け,進捗確認を行った.
全4個のプロジェクト全てが設立された.
2022年度春学期はBCPレベルが2以下を推移している点と教室での活動の許可が下りた点を考慮し対面での活動を推奨し,
三つの班が対面で活動を行った.
2022年度は半期プロジェクトと通年プロジェクトのどちらも設立し,
プロジェクト発表会で通年のプロジェクトは中間報告書と途中成果を発表することとした.

以下に2022年度春学期に活動していたプロジェクトの一覧を示す.

通年プロジェクト
\begin{itemize}
  \item DTM班
  \item Unity班
\end{itemize}

半期プロジェクト
\begin{itemize}
  \item LTプレゼン班
  \item Linux班
\end{itemize}

プロジェクト活動の総括は以下の六つに分けて行う.

\begin{itemize}
  \item 目標の総括
  \item プロジェクトの内容
  \item 週報
  \item 報告書
  \item 追い込み合宿
  \item プロジェクト発表会
\end{itemize}

\subsubsection*{目標の総括}
2021年度春学期の目標は以下の三つであった.

\begin{itemize}
  \item 活動を通して技術力の向上を図る
  \item 個人のみならずグループ活動としての経験を得る
  \item 活動によって得られた成果を本会Webサイトを通して公開する
\end{itemize}

これらを踏まえた総括を以下に記す.

活動を通して技術力の向上を図るに関しては,
行った活動内では行われたが,そもそも活動数が少なかったり参加率が悪かったりしたため十分できたとは言い難い.
この点の原因としては対面での活動は時間的な制約のみだけではなく地理的な制約が大きく関わっていることや,
また,時間的制約においても比較的集まりやすい19時以降の対面活動が出来ないことが挙げられる.

集団行動の重要性を学ぶに関しては,
オンラインでの参加率は従来通り高いと思われるが,対面での参加率は地理的な制約や19時以降の活動ができないことから
低いと思われる.

得られた成果を本会Webサイトを通して公開するに関しては,
渉外局長の退会が5月にあり,また十分な成果を得られなかった為,
成果物のWebサイトを通しての公開や公開の枠組み作りは行われなかった.

\subsubsection*{プロジェクトの内容}
プロジェクトの内容については,全ての班において適切であった.

\subsubsection*{週報}
週報のGoogleフォームは準備したが十分な告知・催促が行われなかったため2件のみの提出に留まり,
プロジェクト活動の状態を把握するのには十分ではなかった.

\subsubsection*{報告書}
半期プロジェクトは報告書の提出をもってプロジェクト終了とし,通年プロジェクトは途中成果を示すこととする.
報告書の必須項目を以下に示す.

\begin{itemize}
  \item 活動動機,目的
  \item 活動内容
  \item 活動結果
  \item 考察
  \item 参考文献
\end{itemize}

上記にも示してある通り活動が十分ではなかったため,報告書を提出できた班は存在しなかった.

\subsubsection*{追い込み合宿}
上記にも示してある通り活動が十分ではなかったため,追い込み合宿をした班は存在しなかった.


\subsubsection*{プロジェクト発表会}
上記にも示してある通り活動が十分ではなかったため,プロジェクト発表会をした班は存在しなかった.

